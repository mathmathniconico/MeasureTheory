% ===========================================
% Project:MeasureTheory
% Filename:root.tex
% Author:mathmathniconico
% ChangeLog:
%	YYYY/MM/DD
%	2018/04/01	執筆開始
%	2018/04/17	0.4.0	「半加法族と測度」を追加
%	2018/04/20	0.5.0	「積測度空間」を追加
%	2018/05/08	0.5.1	1.1.2節の最後の命題は誤りなのでコメントアウト。1.5.1節の最初の命題の条件を緩和。
%				0.6.0	「位相空間と可測空間」を追加
%	2018/05/15	0.6.5	ファイルを分割
%				0.7.0	「ルベーグ-スティルチェス測度」を追加
% 
% ===========================================

% クラスファイル
\documentclass[a4paper, 11pt, report]{ltjsbook}

% ファイル分割
\usepackage{subfiles}

% ページレイアウト
\usepackage{footnote}

% フォント
\usepackage{luatexja}
\usepackage[no-math]{fontspec}
	% 欧文フォント用の設定
\setmainfont[Ligatures=TeX]{Libertinus Serif}
\setsansfont[Ligatures=TeX]{Libertinus Sans}
	% 和文フォント用の設定
\usepackage[match]{luatexja-fontspec}
\usepackage[yu-win]{luatexja-preset}	% 游フォントのプリセット

% 数学
\usepackage{amsmath, amssymb}
\usepackage{mathrsfs}			% 筆記体 $ \mathscr{ } $
\usepackage{stmaryrd}			% 二重鍵括弧 $ \llbracket, \rrbracket $

% グラフィック
\usepackage[svgnames]{xcolor}	% 色を名前で使う
\usepackage{tikz}
\usetikzlibrary{cd}			% 可換図式
\usetikzlibrary{calc}		% 計算
\usepackage{tcolorbox}
\tcbuselibrary{skins, breakable, xparse, theorems}

% 色定義
\providecolorset{HTML}{colsTitlepage}{}{Background,ede7e1;Sidebar,E87B51;Title,5F4F4F;Under,9F9D39;White,FFFFFF}

% 定理型環境の定義
\newtcbtheorem[number within=section]{Thm}{定理}{enhanced, parbox=false, before skip=5mm,
	attach boxed title to top left={xshift=3mm, yshift*=-\tcboxedtitleheight/2},
	boxed title style={sharp corners},
	fonttitle=\bfseries,
	colbacktitle=Maroon,
	colframe=Maroon,
	sharpish corners,
	colback=White,
}{th}
\newtcbtheorem[use counter from=Thm]{Prop}{命題}{enhanced, parbox=false, before skip=5mm,
	attach boxed title to top left={xshift=3mm, yshift*=-\tcboxedtitleheight/2},
	boxed title style={sharp corners},
	fonttitle=\bfseries,
	colbacktitle=DarkGreen,
	colframe=DarkGreen,
	sharpish corners,
	colback=White,
}{pr}
\newtcbtheorem[use counter from=Thm]{Lem}{補題}{enhanced, parbox=false, before skip=5mm,
	attach boxed title to top left={xshift=3mm, yshift*=-\tcboxedtitleheight/2},
	boxed title style={sharp corners},
	fonttitle=\bfseries,
	colbacktitle=Grey,
	colframe=Grey,
	sharpish corners,
	colback=White,
}{lm}
\newtcbtheorem[use counter from=Thm]{Cor}{系}{enhanced, parbox=false, before skip=5mm,
	attach boxed title to top left={xshift=3mm, yshift*=-\tcboxedtitleheight/2},
	boxed title style={sharp corners},
	fonttitle=\bfseries,
	colbacktitle=DarkViolet,
	colframe=DarkViolet,
	sharpish corners,
	colback=White,
}{co}
\newtcbtheorem[use counter from=Thm]{Def}{定義}{enhanced, parbox=false, before skip=5mm,
	attach boxed title to top left={xshift=3mm, yshift*=-\tcboxedtitleheight/2},
	boxed title style={sharp corners},
	fonttitle=\bfseries,
	colbacktitle=DarkBlue,
	colframe=DarkBlue,
	sharpish corners,
	colback=White,
}{df}
\newtcbtheorem[use counter from=Thm]{Rem}{注意}{enhanced, parbox=false, before skip=5mm,
	attach boxed title to top left={xshift=3mm, yshift*=-\tcboxedtitleheight/2},
	boxed title style={sharp corners},
	fonttitle=\bfseries,
	colbacktitle=Goldenrod,
	colframe=Goldenrod,
	sharpish corners,
	colback=White,
}{rk}
\newtcbtheorem[use counter from=Thm]{Exc}{演習}{enhanced, parbox=false, before skip=5mm,
	attach boxed title to top left={xshift=3mm, yshift*=-\tcboxedtitleheight/2},
	boxed title style={sharp corners},
	fonttitle=\bfseries,
	colbacktitle=Black,
	colframe=Black,
	sharpish corners,
	colback=White,
}{xc}
\newtcolorbox{proof}{enhanced, breakable, parbox=false, before skip=5mm,
	arc=2mm,
	borderline={0.5mm}{0mm}{Orange!15!White},
	borderline={0.5mm}{0mm}{Orange!50!White, dashed},
	frame hidden,
	colback=White
}


% 箇条書き
\usepackage{enumitem}
\newlist{EnumCond}{enumerate}{2}
\setlist[EnumCond]{topsep=4pt}
\setlist[EnumCond, 1]{ label=(\arabic*) }
\setlist[EnumCond, 2]{ label=(\roman*) }
\newlist{EnumEquiv}{enumerate}{1}
\setlist[EnumEquiv]{topsep=4pt}
\setlist[EnumEquiv, 1]{ label=(\alph*) }

% 表
\usepackage{booktabs}

% 目次
\setcounter{tocdepth}{1}	% sectionまで

% 参考文献
\renewcommand{\bibname}{参考文献}

% 索引
%\usepackage{makeidx}
%\renewcommand{\seename}{cf.}
%\renewcommand{\alsoname}{cf.}
%\makeindex

% 記号等
\newcommand{\lt}{<}
\renewcommand{\gt}{>} % こっちはあるのか、ややこしいな

\begin{document}
% 表紙
\begin{titlepage}
\begin{tikzpicture}[remember picture, overlay]
	% 背景
	\fill [colsTitlepageBackground] (current page.south west) rectangle (current page.north east);
	% アンダー
	\fill [colsTitlepageUnder] ($(current page.south west) + (0, 30mm)$) rectangle ($(current page.south east) + (0, 42mm)$);
	\node [anchor = south east,
			align = flush right,
			] at ($(current page.south east) + (-10mm, 31mm)$)
		{\fontsize{30pt}{30pt}\color{colsTitlepageWhite}\textgt{version 0.7.0}};
	% 著者
	\node [anchor = south east,
			align = flush right,
			] at ($(current page.south east) + (-10mm, 45mm)$)
		{\fontsize{40pt}{40pt}\color{colsTitlepageUnder}\textgt{mathmathniconico}};
	% サイドバー
	\node [fill = colsTitlepageSidebar,
			anchor = north west,
			minimum height = 40mm,
			minimum width = \paperheight,
			align = flush right,
			text width = \paperheight-40mm,
			rotate = 90
			] at ($(current page.south west) + (20mm, 0)$)
		{\fontsize{80pt}{80pt}\color{colsTitlepageSidebar!50!colsTitlepageWhite}\textgt{Measure Theory}};
	% タイトル
	\node [anchor = north east,
			align = flush right,
			] at ($(current page.north east) + (-15mm, -60mm)$)
		{\fontsize{60pt}{60pt}\color{colsTitlepageTitle}\textgt{測度論ノート}};
\end{tikzpicture}
\end{titlepage}

\thispagestyle{empty}
\vspace*{\stretch{1}}
この作品は、クリエイティブ・コモンズの 表示 - 非営利 - 継承 4.0 国際 ライセンスで提供されています。
ライセンスの写しをご覧になるには、 http://creativecommons.org/licenses/by-nc-sa/4.0/ をご覧頂くか、
Creative Commons, PO Box 1866, Mountain View, CA 94042, USA までお手紙をお送りください。

mathmathniconico(http://arxiv.hatenablog.com/, https://mathtod.online/@mathmathniconico)
\clearpage

\tableofcontents

\chapter{測度論の基礎}
\section{可測空間と測度空間}
\subfile{sections/MeasurableSpace}
\clearpage

\section{外測度と測度}
\subfile{sections/OuterMeasure}
\clearpage

\section{有限加法族と測度}
\subfile{sections/FinitelyAdditive}
\clearpage

\section{半加法族と測度}
\subfile{sections/Semi-additive}
\clearpage

\section{積測度空間と測度の一致}
\subfile{sections/ProductMeasure}
\clearpage



\chapter{位相構造と測度論}
\section{ボレル集合族}
\subfile{sections/BorelMeasurability}
\clearpage

\section{ルベーグ-スティルチェス測度}
\subfile{sections/LebesgueStieltjes}
\clearpage



\end{document}

%\begin{thebibliography}{99}
%\bibitem[Atiyah-MacDonald]{Atiyah-MacDonald} M.F.Atiyah, I.G.MacDonald. Introduction to Commutative Algebra.
%\end{thebibliography}

%\printindex

\end{document}
