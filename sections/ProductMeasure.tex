% ===========================================
% Project:MeasureTheory
% Filename:ProductMeasure.tex
% Author:mathmathniconico
% ChangeLog:
%	YYYY/MM/DD
%	2018/05/15	1.0.0	ファイルを分割
% 
% ===========================================

\documentclass[../root.tex]{subfiles}

\begin{document}
\subsection{積可測空間上の測度}
測度空間$ ( S, \mathscr{A}, \mu_{S} ) $及び$ ( T, \mathscr{B}, \mu_{T} ) $に対して、
積可測空間$ ( S\times T, \mathscr{A}\otimes\mathscr{B} ) $上の測度を構成しよう。積$ \sigma $-加法族は
\[ \mathscr{A}\otimes\mathscr{B}=\sigma\lbrack \mathscr{A}\times T\cup S\times\mathscr{B} \rbrack \]
と定義されるが、基となる集合族$ \mathscr{A}\times T\cup S\times\mathscr{B} $はやや扱い難い。

\begin{Prop}{}{}
$ S\in\mathscr{G}\subset 2^{S}, T\in\mathscr{H}\subset 2^{T} $とする。このとき
\[ \sigma\lbrack \mathscr{G}\times T\cup S\times\mathscr{H} \rbrack = \sigma\lbrack \mathscr{G}\times\mathscr{H} \rbrack \]
が成り立つ。特に$ ( S, \mathscr{A} ), ( T, \mathscr{B} ) $が可測空間のとき
\[ \sigma\lbrack \mathscr{A}\times T\cup S\times\mathscr{B} \rbrack = \sigma\lbrack \mathscr{A}\times\mathscr{B} \rbrack \]
が成り立つ。
\end{Prop}

\begin{proof}
(証明)$ S\in\mathscr{G}, T\in\mathscr{H} $より$ \mathscr{G}\times T\cup S\times\mathscr{H} \subset \mathscr{G}\times\mathscr{H} $である。
故に左辺は右辺を含む。$ G\times H\in\mathscr{G}\times\mathscr{H} $を取る。このとき
\[ G\times H=G\times T\cap S\times H\in\sigma\lbrack \mathscr{G}\times T\cup S\times\mathscr{H} \rbrack \]
より、逆も成り立つ。$ \square $
\end{proof}

$ \mathscr{A}\times\mathscr{B} $は次の意味で都合が良い。

\begin{Prop}{}{}
$ ( S, \mathscr{A} ), ( T, \mathscr{B} ) $は可測空間とする。このとき$ \mathscr{A}\times\mathscr{B} $は半加法族である。
\end{Prop}

\begin{proof}
(証明)まず$ \emptyset=\emptyset\times\emptyset\in\mathscr{A}\times\mathscr{B} $である。次に$ A\times B, C\times D\in\mathscr{A}\times\mathscr{B} $に対し、
\[ A\times B\cap C\times D=( A\cap C )\times( B\times D )\in\mathscr{A}\times\mathscr{B} \]
である。また
\[ ( S\times T )\backslash( A\times B )=( S\backslash A )\times B\sqcup( S\backslash A )\times( T\backslash B )\sqcup( A\times( T\backslash B ) \]
より、$ \mathscr{A}\times\mathscr{B} $の元の非交叉有限和で書ける。従って$ \mathscr{A}\times\mathscr{B} $は半加法族である。$ \square $
\end{proof}

さて、$ \mathscr{A}\otimes\mathscr{B}=\sigma\lbrack \mathscr{A}\times\mathscr{B} \rbrack $上の測度を構成するために拡張定理を用いることを考えたい。
そのためにはまず、半加法族$ \mathscr{A}\times\mathscr{B} $上の前測度を与えなければならない。

\begin{Lem}{}{}
$ ( S, \mathscr{A}, \mu_{S} ), ( T, \mathscr{B}, \mu_{T} ) $を測度空間とする。
このとき集合函数$ \mu\colon\mathscr{A}\times\mathscr{B}\rightarrow\lbrack 0, \infty \rbrack $を、$ A\times B\in\mathscr{A}\times\mathscr{B} $に対し
\[ \mu( A\times B ):=\mu_{S}( A )\mu_{T}( B ) \]
で定めると、$ \mu $は半加法族$ \mathscr{A}\times\mathscr{B} $上の前測度となる。
\end{Lem}

\begin{proof}
(証明)正値であることは明らかなので、有限加法的であることを示す。$ A_{1}\times B_{1}, \dotsc, A_{n}\times B_{n}\in\mathscr{A}\times\mathscr{B} $に対し、
$ \bigsqcup_{i=1}^{n}A_{i}\times B_{i}=A\times B\in\mathscr{A}\times\mathscr{B} $であるとする。このとき
\[ \mu( A\times B )=\sum_{i=1}^{n}\mu( A_{i}\times B_{i} ) \]
が成り立つことを$ n $に関する帰納法で示そう。

$ n=1 $のときは$ \mu( A\times B )=\mu( A_{1}\times B_{1} ) $より正しい。

$ n=k $に対して成り立つとして、$ n=k+1 $を考える。このとき
\[ ( A_{k+1}\times B_{k+1} )\sqcup\bigsqcup_{i=1}^{k}( A_{i}\times B_{i} )=A\times B \]
だから、
\[ (A\backslash A_{k+1} )\times( B\backslash B_{k+1} )=\bigsqcup_{i=1}^{k}( A_{i}\backslash A_{k+1} )\times( B_{i}\backslash B_{k+1} ) \]
が成り立つ。帰納法の仮定より
\[ \mu_{S}( A\backslash A_{k+1} )\mu_{T}( B\backslash B_{k+1} )=\sum_{i=1}^{k}\mu_{S}( A_{i}\backslash A_{k+1} )\mu_{T}( B_{i}\backslash B_{k+1} ) \]
が成り立つ。同様にして
\begin{align*}
\mu_{S}( A\backslash A_{k+1} )\mu_{T}( B\cap B_{k+1} ) &= \sum_{i=1}^{k}\mu_{S}( A_{i}\backslash A_{k+1} )\mu_{T}( B_{i}\cap B_{k+1} ) \\
\mu_{S}( A\cap A_{k+1} )\mu_{T}( B\backslash B_{k+1} ) &= \sum_{i=1}^{k}\mu_{S}( A_{i}\cap A_{k+1} )\mu_{T}( B_{i}\backslash B_{k+1} )
\end{align*}
も成り立つ。ここで$ i=1, \dotsc, k $について$ \mu_{S}( A_{i}\cap A_{k+1} )\mu_{T}( B_{i}\cap B_{k+1} )\gt 0 $と仮定すると、
\[ ( A_{i}\cap A_{k+1} )\times( B_{i}\cap B_{k+1} )=( A_{i}\times B_{i} )\cap( A_{k+1}\times B_{k+1} )\neq\emptyset \]
となり矛盾する。故に$ \mu_{S}( A_{i}\cap A_{k+1} )\mu_{T}( B_{i}\cap B_{k+1} )=0 $であり、
\begin{align*} \mu( A\times B ) &= \mu_{S}( A )\mu_{T}( B ) \\
&= ( \mu_{S}( A\backslash A_{k+1} )+\mu_{S}( A\cap A_{k+1} ) )( \mu_{T}( B\backslash B_{k+1} )+\mu_{T}( B\cap B_{k+1} ) ) \\
&=\sum_{i=1}^{k}( \mu_{S}( A_{i}\backslash A_{k+1} )+\mu_{S}( A_{i}\cap A_{k+1} ) )( \mu_{T}( B_{i}\backslash B_{k+1} )+\mu_{T}( B_{i}\cap B_{k+1} ) ) \\
&+\mu_{S}( A_{k+1} )\mu_{T}( B_{k+1} ) \\ &= \sum_{i=1}^{k+1}\mu_{S}( A_{i} )\mu_{T}( B_{i} )
\end{align*}
を得る。$ \square $
\end{proof}

\begin{Thm}{}{}
上記補題の$ \mu $は弱可算劣加法的である。
\end{Thm}

\begin{proof}
(証明)集合$ X $上の可算被覆$ \mathscr{C}\subset 2^{X} $について考える。
すなわち$ \mathscr{C}=\lbrace C_{1}, C_{2}, \dotsc \rbrace $は$ X=\bigcup_{n\in\mathbb{N}}C_{n} $を満たすとする。
写像$ f\colon X\rightarrow 2^{\mathbb{N}} $を$ x\in X $に対し、$ x\in C_{n} $なら$ f( x )_{n}:=1 $、$ x\notin C_{n} $なら$ f( x )_{n}:=0 $により定める。このとき$ f $は単射であり、
\[ f^{-1}( a )=\bigcap_{a_{i}=1}C_{i}\cap\bigcap_{a_{i}=0}( X\backslash C_{i} ) \]
が成り立つ。特に$ f^{-1}( 0 )=0 $であり、$ X=\bigsqcup_{a\in2^{\mathbb{N}}}f^{-1}( a ) $が成り立つ。

互いに素な$ \lbrace A_{n}\times B_{n} \rbrace\subset\mathscr{A}\times\mathscr{B} $について、
$ \bigsqcup_{n\in\mathbb{N}}A_{n}\times B_{n}=A\times B\in\mathscr{A}\times\mathscr{B} $であるとする。
$ \bigcup_{n\in\mathbb{N}}A_{n}=A, \bigcup_{n\in\mathbb{N}}B_{n}=B $より、
写像$ f\colon A\rightarrow 2^{\mathbb{N}}, g\colon B\rightarrow 2^{\mathbb{N}} $を上記のように定めることができる。
このとき$ \mathscr{A}, \mathscr{B} $は$ \sigma $-加法族だから、$ f^{-1}( a )\in\mathscr{A}, g^{-1}( b )\in\mathscr{B} $が成り立つ。$ \mu_{S}, \mu_{T} $は測度だから可算加法的、つまり
\begin{align*}
\mu( A\times B ) &= \mu_{S}( A )\mu_{T}( B ) \\
&= \left( \sum_{a\in 2^{\mathbb{N}}}\mu_{S}( f^{-1}( a ) ) \right)\left( \sum_{b\in 2^{\mathbb{N}}}\mu_{T}( g^{-1}( b ) ) \right) \\
&=\sum_{a, b}\mu_{S}( f^{-1}( a ) )\mu_{T}( g^{-1}( b ) ) \\
&= \sum_{a, b}\mu( f^{-1}( a )\times g^{-1}( b ) )
\end{align*}
が成り立つ。ところで$ ( x, y )\in A\times B $について、ある$ n $が存在して$ ( x, y )\in A_{n}\times B_{n} $である。
このとき$ f( x )_{n}=g( y )_{n}=1 $であるから、任意の$ n $について$ a_{n}\neq b_{n} $となる$ a, b $について$ f^{-1}( a )\times g^{-1}( b )=\emptyset $となる。従って
\begin{align*} \sum_{a, b}\mu( f^{-1}( a )\times g^{-1}( b ) ) &\le \sum_{n\in\mathbb{N}}\sum_{a_{n}=1}\sum_{b_{n}=1}\mu( f^{-1}( a )\times g^{-1}( b ) ) \\
&=\sum_{n\in\mathbb{N}}\sum_{a_{n}=1}\sum_{b_{n}=1}\mu_{S}( f^{-1}( a ) )\mu_{T}( g^{-1}( b ) ) \\ &=\sum_{n\in\mathbb{N}}\mu_{S}( A_{n} )\mu_{T}( B_{n} ) \\
&= \sum_{n\in\mathbb{N}}\mu( A_{n}\times B_{n} )
\end{align*}
が分かる。以上より$ \mu $が弱可算劣加法的であることが示された。$ \square $
\end{proof}

従って拡張定理より$ \mu $の拡張となる$ \mathscr{A}\otimes\mathscr{B}=\sigma\lbrack \mathscr{A}\times\mathscr{B} \rbrack $上の測度が存在する。
これをもって測度空間$ ( S, \mathscr{A}, \mu_{S} ), ( T, \mathscr{B}, \mu_{T} ) $の積としたいのだが、実は拡張は一意でない。




\subsection{ディンキン族}
\begin{Def}{}{}
集合$ S $において$ \mathscr{D}\subset 2^{S} $が次の3条件を満たすとき、$ \mathscr{D} $は$ S $上のディンキン族であるという。
\begin{EnumCond}
\item $ S\in\mathscr{D} $である。
\item $ A, B\in\mathscr{D}, A\subset B $なら$ B\backslash A\in\mathscr{D} $である。
\item $ \lbrace D_{n} \rbrace_{n\in\mathbb{N}}\subset\mathscr{D} $が単調増大列($ D_{1}\subset D_{2}\subset\dotsm $)なら$ \bigcup_{n\in\mathbb{N}}D_{n}\in\mathscr{D} $である。
\end{EnumCond}
\end{Def}

明らかに$ \emptyset\in\mathscr{D} $である。また$ \sigma $-加法族はディンキン族である。

ディンキン族も任意の交叉でディンキン族となるため、生成を考えることができる。

\begin{Def}{}{}
$ \mathscr{G}\subset 2^{S} $について、$ \mathscr{S} $を含むディンキン族全体の交叉を$ D\lbrack \mathscr{G} \rbrack_{S} $、
あるいは単に$ D\lbrack \mathscr{G} \rbrack $で記し、$ \mathscr{G} $により$ S $上で生成されたディンキン族と呼ぶ。
\end{Def}

$ D\lbrack \mathscr{G} \rbrack $は$ \mathscr{G} $を含む最小のディンキン族である。

\begin{Prop}{}{}
$ \mathscr{G}\subset 2^{S} $とする。$ A\in D\lbrack \mathscr{G} \rbrack $に対して
\[ \mathscr{D}_{A}:=\lbrace B\in D\lbrack \mathscr{G} \rbrack : A\cap B\in D\lbrack \mathscr{G} \rbrack \rbrace \]
と定めると$ \mathscr{D}_{A} $はディンキン族となる。
\end{Prop}

\begin{proof}
(証明)$ A\cap X= A\in D\lbrack \mathscr{G} \rbrack $より$ X\in\mathscr{D}_{A} $である。
$ B, C\in D\lbrack \mathscr{G} \rbrack, B\subset C $に対して$ A\cap ( C\backslash B )=( A\cap C )\backslash( A\cap B ) $となる。
ここで$ A\cap B, A\cap C\in D\lbrack \mathscr{G} \rbrack $は$ A\cap B\subset A\cap C $を満たすので、$ C\backslash B\in\mathscr{D}_{A} $が分かる。
単調増大列$ \lbrace B_{n} \rbrace_{n\in\mathbb{N}}\subset D\lbrack \mathscr{G} \rbrack $を取る。
$ A\cap\bigcup_{n\in\mathbb{N}}B_{n}=\bigcup_{n\in\mathbb{N}}( A\cap B_{n} ) $だが、
これは単調増大列$ \lbrace A\cap B_{n} \rbrace_{n\in\mathbb{N}}\subset D\lbrack \mathscr{G} \rbrack $の極限で表せる。
故に$ \bigcup_{n\in\mathbb{N}}B_{n}\in\mathscr{D}_{A} $も従う。$ \square $
\end{proof}

\begin{Lem}{ディンキンの補題}{}
$ \mathscr{G}\subset 2^{S} $は有限交叉で閉じるとする。すなわち$ G_{1}, \dotsc, G_{n}\in\mathscr{G} $について、
\[ \bigcap_{i=1}^{n}G_{i}\in\mathscr{G} \]
であるとする。このとき$ D\lbrack \mathscr{G} \rbrack=\sigma\lbrack \mathscr{G} \rbrack $が成り立つ。
\end{Lem}

\begin{proof}
(証明)$ \sigma $-加法族はディンキン族であるから、最小性より$ D\lbrack \mathscr{G} \rbrack\subset\sigma\lbrack \mathscr{G} \rbrack $である。
逆は$ D\lbrack \mathscr{G} \rbrack $が$ \sigma $-加法族であることを示せばよい。

$ A\in\mathscr{G} $とする。
任意の$ G\in\mathscr{G} $に対し仮定より$ A\cap G\in\mathscr{G}\subset D\lbrack \mathscr{G} \rbrack $であるから$ \mathscr{G}\subset\mathscr{D}_{A} $が分かる。
$ \mathscr{D}_{A} $はディンキン族だから最小性より$ D\lbrack \mathscr{G} \rbrack\subset\mathscr{D}_{A} $を得る。
逆も定義より明らかなので、$ A\in\mathscr{G} $は$ D\lbrack \mathscr{G} \rbrack=\mathscr{D}_{A} $を満たす。

ここで
\[ \mathscr{D}:=\lbrace A\in D\lbrack \mathscr{G} \rbrack : \mathscr{D}_{A}=D\lbrack \mathscr{G} \rbrack \rbrace \]
と定める。上の議論より$ \mathscr{G}\subset\mathscr{D} $となる。そこで$ \mathscr{D} $が$ S $上のディンキン族となることを示そう。
$ \mathscr{D}_{X}=D\lbrack \mathscr{G} \rbrack $より$ X\in\mathscr{D} $である。$ A, B\in\mathscr{D}, A\subset B $とする。
$ G\in\mathscr{G} $に対し$ G\cap( B\backslash A )=( G\cap B )\backslash( G\cap A )\in D\lbrack \mathscr{G} \rbrack $が成り立つ。
故に$ \mathscr{D}_{B\backslash A} $は$ \mathscr{G} $を含むディンキン族となり$ \mathscr{D}_{B\backslash A}=D\lbrack \mathscr{G} \rbrack $を満たす。
同様に単調増大列$ \lbrace A_{n} \rbrace_{n\in\mathbb{N}}\subset\mathscr{D} $を取れば、
$ G\in\mathscr{G} $に対し$ ( \bigcup_{n\in\mathbb{N}} )\cap G=\bigcup_{n\in\mathbb{N}}( G\cap A_{n} ) $が成り立つ。
これは単調増大列$ \lbrace G\cap A_{n} \rbrace_{n\in\mathbb{N}}\subset D\lbrack \mathscr{G} \rbrack $の極限だから
結局$ D\lbrack \mathscr{G} \rbrack = \mathscr{D}_{\bigcup_{n\in\mathbb{N}}} $を得る。
以上により$ \mathscr{D} $は$ S $上のディンキン族となる。特に$ \mathscr{G} $を含むことから$ \mathscr{D}=D\lbrack \mathscr{G} \rbrack $が従う。

最後に$ \mathscr{D} $が$ \sigma $-加法族であることを示そう。$ A, B\in\mathscr{D} $に対し、$ A\backslash B=A\backslash( A\cap B )\in\mathscr{D} $である。
特に$ \emptyset=X\backslash X\in\mathscr{D} $となる。また$ A\cup B=X\backslash( ( X\backslash A )\cap( X\backslash B ) )\in\mathscr{D} $も分かる。
$ \lbrace A_{n} \rbrace_{n\in\mathbb{N}}\subset\mathscr{D} $について、$ B_{n}=\bigcup_{i=1}^{n}A_{i} $と定めれば
$ \lbrace B_{n} \rbrace_{n\in\mathbb{N}}\subset\mathscr{D} $は単調増大列となる。
従って$ \bigcup_{n\in\mathbb{N}}A_{n}=\bigcup_{n\in\mathbb{N}}B_{n}\in\mathscr{D} $となる。$ \square $
\end{proof}




\subsection{測度の一致}
\begin{Def}{}{}
単調な集合函数$ \mu\colon\mathscr{G}\rightarrow\lbrack 0, \infty \rbrack $に対し以下を定める。
\begin{EnumCond}
\item $ S\in\mathscr{G} $であり、$ \mu( G )\lt\infty $のとき$ \mu $は有限(finite)であるという。
\item ある$ \lbrace G_{n} \rbrace\subset\mathscr{G} $が存在して$ G_{n}\nearrow S, \mu( G_{n} )\lt\infty $を満たすとき$ \mu $は$ \sigma $-有限であるという。
\end{EnumCond}
\end{Def}

\begin{Prop}{}{}
可測空間$ ( S, \mathscr{A} ) $上の有限な測度$ \mu_{1}, \mu_{2} $に対し、$ \mu_{1}( S )=\mu_{2}( S ) $なら
\[ \mathscr{D}:=\lbrace D\in\mathscr{A} : \mu_{1}( D )=\mu_{2}( D ) \rbrace \]
はディンキン族である。
\end{Prop}

\begin{proof}
(証明)定義より$ S\in\mathscr{D} $である。$ A, B\in\mathscr{D}, A\subset B $とする。$ \mu_{j} $は有限な測度だから$ \mu_{j}( B\backslash A )=\mu_{j}( B )-\mu_{j}( A ) $となる。
故に$ B\backslash A\in\mathscr{D} $となる。また単調増大列$ \lbrace A_{n} \rbrace\subset\mathscr{D} $に対し、$ A_{0}:=\emptyset, B_{n}:=A_{n}\backslash A_{n-1} $と定めれば
\[ \mu_{j}\left( \bigcup_{n\in\mathbb{N}} \right)=\mu_{j}\left( \bigsqcup_{n\in\mathbb{N}}B_{n} \right)=\sum_{n\in\mathbb{N}}\mu_{j}( A_{n}\backslash A_{n-1} ) \]
が成り立つ。$ A_{n}\backslash A_{n-1}\in\mathscr{D} $より$ \bigcup_{n\in\mathbb{N}}A_{n}\in\mathscr{D} $が従う。$ \square $
\end{proof}

\begin{Thm}{}{}
$ \mathscr{G}\subset 2^{S} $は有限交叉で閉じるとする。$ \sigma\lbrack \mathscr{G} \rbrack $上の測度$ \mu_{1}, \mu_{2} $は、$ \mathscr{G} $上で一致し、更に
$ \mu_{0}:=\mu_{j}|_{\mathscr{G}} $は$ \sigma $-有限とする。このとき$ \mu_{1}=\mu_{2} $が成り立つ。
\end{Thm}

\begin{proof}
(証明)単調増大な$ \lbrace G_{n} \rbrace\subset\mathscr{G} $を、$ G_{n}\nearrow S, \mu_{0}( G_{n} )\lt\infty $を満たすように取る。
$ A\in\sigma\lbrack \mathscr{G} \rbrack $に対して$ A\cap G_{n}\nearrow A $であるから、増大列連続性より$ \mu_{j}( A )=\lim_{n\in\mathbb{N}}\mu( A\cap G_{n} ) $が成り立つ。

$ A\in\sigma\lbrack \mathscr{G} \rbrack $に対して$ \widetilde{\mu}_{j}( A ):=\mu_{j}( A\cap G_{n} ) $と定めると、
$ \widetilde{\mu}_{j} $は$ \sigma\lbrack \mathscr{G} \rbrack $上の有限な測度となる。ここで
\[ \mathscr{D}_{n}:=\lbrace A\in\sigma\lbrack \mathscr{G} \rbrack : \widetilde{\mu}_{1}( A )=\widetilde{\mu}_{2}( A ) \rbrace \]
と置くと、先の命題よりこれはディンキン族となる。ディンキンの補題より$ \sigma\lbrack \mathscr{G} \rbrack=D\lbrack \mathscr{G} \rbrack $であるから、
最小性より$ \mathscr{D}_{n}=\sigma\lbrack \mathscr{G} \rbrack $となる。よって$ \mu_{1}( A\cap G_{n} )=\mu_{2}( A\cap G_{n} ) $であるから、$ \mu_{1}=\mu_{2} $が従う。$ \square $
\end{proof}

\begin{Cor}{}{}
$ ( S, \mathscr{A}, \mu_{S} ), ( T, \mathscr{B}, \mu_{T} ) $を測度空間とする。$ \mu_{S}, \mu_{T} $が$ \sigma $-有限であるとき、
前測度$ \mu\colon\mathscr{A}\times\mathscr{B}\rightarrow\lbrack 0, \infty \rbrack $の拡張は一意的である。
\end{Cor}

\begin{proof}
(証明)$ \lbrace A_{n} \rbrace_{n\in\mathbb{N}}\subset\mathscr{A}, \lbrace B_{n} \rbrace_{n\in\mathbb{N}}\subset\mathscr{B} $として
\begin{align*}
A_{n}&\nearrow S, & B_{n}&\nearrow T, & \mu_{S}( A_{n} ), \mu_{T}( B_{n} )&\lt\infty
\end{align*}
を満たすように取る。ここで$ C_{n}=A_{n}\times B_{n}\in\mathscr{A}\times\mathscr{B} $とすれば
\begin{align*}
C_{n}&\nearrow S\times T, & \mu( C_{n} )&=\mu_{S}( A_{n} )\mu_{T}( B_{n} )\lt\infty
\end{align*}
が成り立つ。従って$ \mu $は$ \sigma $-有限であるため、定理から拡張は一意的であることが分かる。$ \square $
\end{proof}

\begin{Def}{}{}
系において$ \mu $の拡張となる可測空間$ ( S\times T, \mathscr{A}\otimes\mathscr{B} ) $上の測度は一意的に存在する。
これを$ \mu_{S}\otimes\mu_{T} $と記し、$ \mu_{S} $と$ \mu_{T} $の積測度と呼ぶ。このとき$ ( S\times T, \mathscr{A}\otimes\mathscr{B}, \mu_{S}\otimes\mu_{T} ) $を積測度空間と呼ぶ。
\end{Def}

ただし積測度空間は圏における積対象ではない。

\end{document}