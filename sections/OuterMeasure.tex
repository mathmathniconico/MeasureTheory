% ===========================================
% Project:MeasureTheory
% Filename:OuterMeasure.tex
% Author:mathmathniconico
% ChangeLog:
%	YYYY/MM/DD
%	2018/05/15	1.0.0	ファイルを分割
% 
% ===========================================

\documentclass[../root.tex]{subfiles}

\begin{document}
この節より用いる集合函数に関する諸定義を述べておこう。$ S $を空でない集合とする。$ \emptyset\in\mathscr{G}\subset 2^{S} $とし、$ \mu\colon\mathscr{G}\rightarrow\lbrack 0, \infty \rbrack $を集合函数とする。
\begin{EnumCond}
\item $ A, B\in\mathscr{G} $について$ A\subset B \Rightarrow \mu( A )\le\mu( B ) $が成り立つとき$ \mu $は単調(monotone)であるという。
\item $ A_{1}, \dotsc, A_{m}\in\mathscr{G} $に対し、
\[ A:=\bigcup_{i=1}^{m}A_{i}\in\mathscr{G} \Rightarrow \mu( A )\le\sum_{i=1}^{m}\mu( A_{i} ) \]
が成り立つとき$ \mu $は有限劣加法的(finite-subadditive)であるという。
\item $ \lbrace A_{n} \rbrace_{n\in\mathbb{N}}\subset\mathscr{G} $に対し、
\[ A:=\bigcup_{n\in\mathbb{N}} A_{n}\in\mathscr{G} \Rightarrow \mu( A )\le\sum_{n\in\mathbb{N}}\mu( A_{n} ) \]
が成り立つとき$ \mu $は可算劣加法的(countable-subadditive)であるという。
\end{EnumCond}

$ \sigma $-加法族上の測度は上記の性質を全て満たしている。




\subsection{外測度}
\begin{Def}{}{}
集合函数$ \mu\colon 2^{S}\rightarrow\lbrack 0, \infty \rbrack $が正値、単調、可算劣加法的のとき、外測度(outer measure)あるいはカラテオドリ(Caratheodory's)の外測度と呼ぶ。
\end{Def}

測度の概念については、ユークリッド空間$ \mathbb{R}^{n} $上の$ n $次元体積が念頭にあることは言うまでもない。まず矩形に対し、各辺の「長さ」の積をその体積とする。
ここである図形$ A $の体積が知りたければ、十分小さな矩形で覆い、その体積の和を調べ、更にその下限を取ることで「外側の体積」とする。これが外測度を導入する理由である。

以下$ \emptyset\in\mathscr{E}\subset 2^{S} $とする。$ A\subset S $について、$ \mathscr{C}\subset\mathscr{E} $が$ A $の被覆であるとは、
$ \mathscr{C} $が可算集合であり、$ A\subset\bigcup_{C\in\mathscr{C}}C $を満たすこととする。

\begin{Lem}{}{}
$ \emptyset\in\mathscr{E}\subset 2^{S} $とする。$ \mathscr{E} $上の集合函数$ \mu_{0}\colon\mathscr{E}\rightarrow\lbrack 0, \infty \rbrack $は正値とする。このとき集合函数$ \mu\colon 2^{S}\rightarrow\lbrack 0, \infty \rbrack $を
\[ \mu( A ):=\inf\left\lbrace \sum_{C\in\mathscr{C}}\mu_{0}( C ) : \mathscr{C}\subset\mathscr{E}\text{は}A\text{の被覆} \right\rbrace \]
と定める。(ただし$ A $が$ \mathscr{E} $の可算個の元で覆えないときは$ \mu( A ):=\infty $と定める。)このとき$ \mu $は外測度となる。
\end{Lem}

\begin{proof}
(証明)$ \mathscr{C}=\lbrace \emptyset \rbrace $を取れば$ \emptyset $の被覆となるから$ \mu( \emptyset )\le\mu_{0}( \emptyset )=0 $なので正値である。

また$ A\subset B $に対して$ B $の被覆は$ A $の被覆でもあるから単調性も従う。

可算劣加法的であることを示すために$ \lbrace A_{n} \rbrace_{n\in\mathbb{N}}\subset 2^{S} $を取り$ A:=\bigcup_{n\in\mathbb{N}}A_{n} $とする。
正の実数$ \varepsilon\gt 0 $を固定する。このとき$ A_{n} $の被覆$ \mathscr{C}_{n} $を、
\[ \sum_{C\in\mathscr{C}_{n}}\mu_{0}( C )\le\mu( A_{n} )+\frac{\varepsilon}{2^{n}} \]
を満たすように取れる。($ A_{n} $の何れかが$ \mathscr{E} $の可算個の元で覆えないときは$ A $も覆えないため、$ \infty \le \infty $となり主張は正しい。)
このとき$ \mathscr{C}=\bigcup_{n\in\mathbb{N}}\mathscr{C}_{n} $は$ A $の被覆であり、
\begin{align*}
\mu( A ) &\le \sum_{C\in\mathscr{C}}\mu_{0}( C ) \le \sum_{n\in\mathbb{N}}\sum_{C\in\mathscr{C}_{n}}\mu_{0}( C ) \\
&\le \sum_{n\in\mathbb{N}}\left( \mu( A_{n} )+\frac{\varepsilon}{2^{n}} \right) = \sum_{n\in\mathbb{N}}\mu( A_{n} )+\varepsilon
\end{align*}
となる。$ \varepsilon $は任意だから$ \mu( A )\le\sum_{n\in\mathbb{N}}\mu( A_{n} ) $を得る。$ \square $
\end{proof}

一般に$ E\in\mathscr{E} $なら$ \lbrace E \rbrace $が$ E $の被覆となるので$ \mu( E )\le\mu_{0}( E ) $が成り立つ。

補題の方法で定義された外測度を$ \mu_{0}\colon\mathscr{E}\rightarrow\lbrack 0, \infty \rbrack $から誘導された外測度と呼ぶこともある。この値については次の特徴付けがある。

\begin{Prop}{}{}
$ \emptyset\in\mathscr{E}\subset 2^{S} $とする。$ \mu $を$ \mu_{0}\colon\mathscr{E}\rightarrow\lbrack 0, \infty \rbrack $から誘導された外測度とする。このとき$ B\subset S $について
\[ \mu( B )=\inf\lbrace \mu( A ) : A\in\sigma\lbrack \mathscr{E} \rbrack, B\subset A \rbrace \]
が成り立つ。特にこの式を実現する$ A\in\sigma\lbrack \mathscr{E} \rbrack $が存在する。
\end{Prop}

\begin{proof}
(証明)まず$ A\in\sigma\lbrack \mathscr{E} \rbrack $が$ B\subset A $を満たすとする。外測度は単調だから$ \mu( B )\le\mu( A ) $が従う。
特に(左辺)$ \le $(右辺)であり、また逆を示すときには$ \mu( B )\lt\infty $としてよいことが分かる。このとき$ \mathscr{C}\subset\mathscr{E} $を$ B $の被覆とする。
すると$ \sigma $-加法性より$ B\subset\bigcup_{C\in\mathscr{C}}C\in\sigma\lbrack \mathscr{E} \rbrack $であるから、
\[ \inf\lbrace \mu( A ) : A\in\sigma\lbrack \mathscr{E} \rbrack, B\subset A \rbrace \le \mu\left( \bigcup_{C\in\mathscr{C}}C \right) \le \sum_{C\in\mathscr{C}}\mu( C ) \le \sum_{C\in\mathscr{C}}\mu_{0}( C ) \]
を得る。被覆の下限を取れば(左辺)$ \ge $(右辺)も分かる。

今示した式より、$ \lbrace A_{n} \rbrace\subset\sigma\lbrack \mathscr{E} \rbrack $として$ B\subset A_{n} $及び$ \mu( A_{n} )\searrow\mu( B ) $を満たすものが取れる。
このとき$ A:=\bigcap_{n\in\mathbb{N}}A_{n}\in\sigma\lbrack \mathscr{E} \rbrack $について、
$ B\subset A\subset A_{n} $だから単調性より$ \mu( B )\le\mu( A )\le\mu( A_{n} ) $が成り立つ。特に右辺は$ \mu( B ) $へ収束するから$ \mu( A )=\mu( B ) $が従う。$ \square $
\end{proof}




\subsection{外測度による測度の構成}
次の定義を導入したことこそ、カラテオドリの偉大なところであろう。私自身はこの定義についてよく理解していないのだが。

\begin{Def}{}{}
外測度$ \mu $に対し、$ A\subset S $がカラテオドリ可測、あるいは$ \mu $-可測であるとは、任意の$ E\subset S $について$ \mu( E )=\mu( E\cap A )+\mu( E\backslash A ) $が成り立つことをいう。
\end{Def}

根源的な着想はルベーグに依るらしい。ルベーグ自身は$ E $として矩形を考えていた。

ところで$ E=( E\cap A )\cup( E\backslash A ) $であるから、外測度の可算劣加法性より$ \mu( E )\le\mu( E\cap A )+\mu( E\backslash A ) $は常に成り立つ。
つまりカラテオドリ可測であることを示すには$ \mu( E )\ge\mu( E\cap A )+\mu( E\backslash A ) $を示せば十分である。

\begin{Lem}{}{}
$ A, B\subset S $とする。$ A $がカラテオドリ可測なら、任意の$ E\subset S $について
\[ \mu( E\cap( A\cup B ) ) = \mu( E\cap A )+\mu( ( E\backslash A )\cap B ) \]
が成り立つ。特に$ A $と$ B $が互いに素なら
\[ \mu( E\cap( A\sqcup B ) ) = \mu( E\cap A )+\mu( E\cap B ) \]
が成り立つ。
\end{Lem}

\begin{proof}{}{}
(証明)$ A $がカラテオドリ可測なら、任意の$ E\subset S $について
\begin{align*}
\mu( E\cap( A\cup B ) ) &= \mu( ( E\cap( A\cup B ) )\cap A ) + \mu( ( E\cap( A\cup B ) )\backslash A ) \\
&= \mu( E\cap A )+\mu( ( E\backslash A )\cap B )
\end{align*}
が従う。$ \square $
\end{proof}

\begin{Thm}{}{}
外測度$ \mu\colon 2^{S}\rightarrow\lbrack 0, \infty \rbrack $について、$ \mathscr{M}_{\mu} $をカラテオドリ可測な集合全体とする。
このとき$ \mathscr{M}_{\mu} $は$ \sigma $-加法族であり、$ \mu $の$ \mathscr{M}_{\mu} $への制限は可測空間$ ( S, \mathscr{M}_{\mu} ) $上の測度となる。
\end{Thm}

\begin{proof}
(証明)まず$ E\subset S $について$ \mu( E\cap\emptyset )+\mu( E\backslash\emptyset )=\mu( \emptyset )+\mu( E )=\mu( E ) $より$ \emptyset $はカラテオドリ可測である。

また$ A $がカラテオドリ可測なら
\[ \mu( E\cap( S\backslash A ) )+\mu( E\backslash( S\backslash A ) ) = \mu( E\backslash A )+\mu( E\cap A )=\mu( E ) \]
より$ S\backslash A $もカラテオドリ可測となる。

次に$ \mathscr{M}_{\mu} $が有限和に関して閉じていることを述べる。$ A, B $をカラテオドリ可測とすると、補題より任意の$ E\subset S $について
\[ \mu( E\cap( A\cup B ) ) = \mu( E\cap A )+\mu( ( E\backslash A )\cap B ) \]
が成り立つ。一方$ E\backslash( A\cup B )=(E\backslash A )\backslash B $であるから、
\[ \mu( E\cap ( A\cup B ) )+\mu( E\backslash( A\cup B ) ) = \mu( E\cap A )+\mu( ( E\backslash A )\cap B )+\mu( ( E\backslash A )\backslash B ) \]
となる。右辺は$ B $の$ \mu $-可測性より$ \mu( E\cap A )+\mu( E\backslash A) $となり、これは再び$ A $の$ \mu $-可測性より$ \mu( E ) $と一致する。
従って$ A\cup B $はカラテオドリ可測となる。あとはこれを繰り返せば良い。

ここまでの議論で、$ \mathscr{M}_{\mu} $は有限加法族と呼ばれる集合族になることが分かる。(有限加法族についてはまた改めて議論するが、$ \sigma $-加法族の条件で可算和の代わりに有限和としたもの。)
有限加法族$ \mathscr{A} $に対しては、有限交叉や差集合で閉じている。実際$ A, B\in\mathscr{A} $に対して$ S=S\backslash \emptyset\in\mathscr{A} $であり、
$ A\cap B=A\backslash( S\backslash B )\in\mathscr{A} $であり、$ A\backslash B=S\backslash( ( S\backslash A )\cup B )\in\mathscr{A} $である。

ここで$ \mu $の$ \mathscr{M}_{\mu} $への制限を$ \nu $と書くことにする。$ \nu\colon\mathscr{M}_{\mu}\rightarrow\lbrack 0, \infty \rbrack $は有限加法的である。
実際$ A, B\in\mathscr{M}_{\mu} $について、補題より任意の$ E\subset S $について
\[ \mu( E\cap( A\sqcup B ) ) = \mu( E\cap A )+\mu( E\cap B ) \]
が成り立つ。特に$ E= S $と置けば
\[ \nu( A\sqcup B )=\mu( A\sqcup B )=\mu( A )+\mu( B )=\nu( A )+\nu( B ) \]
が分かる。$ \mathscr{M}_{\mu} $は有限加法族なので$ m $個の和を$ 2 $個ずつ考えれば$ \nu $が有限加法的であることも分かる。

さて$ \mathscr{M}_{\mu} $が$ \sigma $-加法族であることを示すために$ \lbrace A_{n} \rbrace\subset\mathscr{M}_{\mu} $を取る。
$ B_{m}:=\bigcup_{n=1}^{m}A_{n} $と置けば、$ B_{m} $はカラテオドリ可測であり$ B_{m}\nearrow\bigcup_{n\in\mathbb{N}}A_{n}=\colon B $である。ここで
\[ C_{1}:=B_{1}, C_{n}:=B_{n}\backslash B_{n-1} \]
と定めれば$ C_{n} $もカラテオドリ可測であり、$ B_{m}=\bigsqcup_{n=1}^{m}C_{n} $と非交叉和で書ける。補題より任意の$ E\subset S $に対して
\[ \mu( E\cap B_{m} )=\sum_{n=1}^{m}\mu( E\cap C_{n} ) \]
が成り立つ。単調性より$ \mu( E\backslash B )\le\mu( E\backslash B_{m} ) $が成り立つので、$ B_{m} $の$ \mu $-可測性より
\[ \mu( E )=\mu( E\cap B_{m} )+\mu( E\backslash B_{m} )\ge\sum_{n=1}^{m}\mu( E\cap C_{n} )+\mu( E\backslash B ) \]
となる。$ m $は任意だから$ \mu( E )\ge\sum_{n\in\mathbb{N}}\mu( E\cap C_{n} )+\mu( E\backslash B ) $となる。ここで
\[ E\cap B=\bigcup_{n\in\mathbb{N}}( E\cap C_{n} ) \]
だから、外測度の可算劣加法性より$ \sum_{n\in\mathbb{N}}\mu( E\cap C_{n} )\ge\mu( E\cap B ) $が成り立つ。
従って$ \mu( E )\ge\mu( E\cap B )+\mu( E\backslash B ) $となり、これは$ B $がカラテオドリ可測であることを意味している。

最後に$ \nu $が可算加法的であることを示すために、互いに素な$ \lbrace A_{n} \rbrace_{n\in\mathbb{N}}\subset\mathscr{M}_{\mu} $を取る。
$ \mathscr{M}_{\mu} $は$ \sigma $-加法族であるから$ A:=\bigsqcup_{n\in\mathbb{N}}A_{n}\in\mathscr{M}_{\mu} $である。外測度の可算劣加法性より
\[ \nu(A)=\mu\left( \bigsqcup_{n\in\mathbb{N}}A_{n} \right)\le\sum_{n\in\mathbb{N}}\mu( A_{n} )=\sum_{n\in\mathbb{N}}\nu( A_{n} ) \]
である。また単調性及び$ \nu $が有限加法的であることから
\[ \nu( A )=\mu\left( \bigsqcup_{n\in\mathbb{N}}A_{n} \right)\ge\mu\left( \bigsqcup_{n=1}^{m}A_{n} \right)=\nu\left( \bigsqcup_{n=1}^{m}A_{n} \right)=\sum_{n=1}^{m}\nu( A_{n} ) \]
が分かる。$ m $は任意だから$ \nu( A )\ge\sum_{n\in\mathbb{N}}\nu( A_{n} ) $を得る。$ \square $
\end{proof}

\end{document}