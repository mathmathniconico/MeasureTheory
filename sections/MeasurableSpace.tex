% ===========================================
% Project:MeasureTheory
% Filename:MeasurableSpace.tex
% Author:mathmathniconico
% ChangeLog:
%	YYYY/MM/DD
%	2018/05/15	1.0.0	ファイルを分割
% 
% ===========================================

\documentclass[../root.tex]{subfiles}

\begin{document}
伊藤清という数学者は、確率とはルベーグ測度のことである、と言ったとか言わなかったとか。
コルモゴロフに始まる測度論を用いた確率論の公理化は数学的に明瞭で、厳密に取り扱うのが面倒なことを除けば疑う余地はない。
最近はゲーム理論的な解釈もあるらしいが、枠組みとしては測度論を使うのが自然だろう。




\subsection{可測空間と圏}
測度論を圏の枠組みで捉えるのであれば、対象としてまず考えられるのは可測空間だろう。

\begin{Def}{}{}
集合$ S $において$ \mathscr{A}\subset 2^{S} $が次の3条件を満たすとき、$ \mathscr{A} $は$ S $上の$ \sigma $-加法族であるという。
\begin{EnumCond}
\item $ \emptyset\in\mathscr{A} $である。
\item $ A\in\mathscr{A} $なら$ S\backslash A\in\mathscr{A} $である。
\item $ A_{n}\in\mathscr{A} $なら$ \bigcup_{n\in\mathbb{N}}A_{n}\in\mathscr{A} $である。
\end{EnumCond}

このとき組$ ( S, \mathscr{A} ) $を可測空間(measurable space)といい、
集合$ A\in\mathscr{A} $は$ \mathscr{A} $-可測、あるいは単に可測である、などという。
\end{Def}

定義に解釈を求めても仕方ないが、可算という視点で物事を捉える、というのが測度論の基本的な考え方とするならば、
有限の立場で議論する論理学(特に命題論理)のアナロジーという意味で、この定義は自然なものだと考えられる。

$ \sigma $-加法族は歴史的な経緯から呼び方が安定しない。他に完全加法族、可算加法族、$ \sigma $-集合代数、$ \sigma $-集合体、$ \sigma $-体、トライブ(tribe)などと呼ばれているようだ。

例えば自明なものとして、$ \lbrace \emptyset, S \rbrace $や$ 2^{S} $などが$ \sigma $-加法族となる。

$ \sigma $-加法族は基本的な集合演算に関して閉じている。
即ち可測な集合の差集合$ A\backslash B $($ A, B\in\mathscr{A} $)や可算交叉$ \bigcap_{n\in\mathbb{N}}A_{n} $($ A_{n}\in\mathscr{A} $)などは全て可測である。

\begin{Prop}{}{}
写像$ f\colon S\rightarrow T $について以下が成り立つ。
\begin{EnumCond}
\item $ ( T, \mathscr{B} ) $が可測空間なら$ f^{-1}( \mathscr{B} ):=\lbrace f^{-1}( B ) : B\in\mathscr{B} \rbrace $は$ S $上の$ \sigma $-加法族となる。
\item $ ( S, \mathscr{A} ) $が可測空間なら$ f( \mathscr{A} ):=\lbrace B\subset T : f^{-1}( B )\in\mathscr{A} \rbrace $は$ T $上の$ \sigma $-加法族となる。
($ \lbrace f( A ) : A\in\mathscr{A} \rbrace $ではないことに注意。)
\end{EnumCond}
\end{Prop}

ひとまず対象については可測空間を考えるとして、次に考えるべきは射であろう。

\begin{Def}{}{}
$ ( S, \mathscr{A} ), ( T, \mathscr{B} ) $を可測空間とする。
写像$ f\colon S\rightarrow T $が$ f^{-1}(\mathscr{B})\subset\mathscr{A} $を満たすとき、
つまり任意の$ B\in\mathscr{B} $について$ f^{-1}( B )\in\mathscr{A} $を満たすとき、
$ f $は$ \mathscr{A}/\mathscr{B} $-可測、あるいは単に可測であるという。
\end{Def}

この定義であれば、$ \mathscr{B} $から$ \mathscr{A} $への写像として定めれば良いのではないかと思うかもしれない。
しかし、あくまで全体空間における写像として考えるのには理由があって、それは「測ることのできない集合」を念頭に入れているからである。
可測でなくても測りたい集合が存在する。このようなものについて議論するために全体空間が必要なのだ。

可測写像の合成は可測であり、恒等写像はもちろん可測である。故に可測空間と可測写像は圏の対象と射を定める。これを圏$ \mathbf{meas} $と記すことにする。




\subsection{$ \sigma $-加法族の生成と可測空間の積}
次に可測空間$ X=( S, \mathscr{A} ) $と可測空間$ Y=( T, \mathscr{B} ) $の対象としての積$ X\prod^{\mathrm{cat}}Y $を考えたい。
射が全体空間における写像であることを考えれば、あるいは集合の圏$ \mathbf{Set} $への忘却函手の存在を考えれば、積対象の全体空間は$ S\times T $で定めるのが妥当だろう。
更に射影$ p\colon X\prod^{\mathrm{cat}}Y\rightarrow X $の存在を考慮すると、
少なくとも$ A\times T, S\times B $($ A\in\mathscr{A}, B\in\mathscr{B} $)の形をした集合は可測となる必要がある。
しかし一般にこれらの集合は$ \sigma $-加法族を成さない。このような場合には生成という方法が有効になる。

まず$ \sigma $-加法族の任意の交叉は$ \sigma $-加法族である。
つまり任意の$ \lambda\in\Lambda $について$ \mathscr{A}_{\lambda} $を$ \sigma $-加法族とすると、
$ \ \bigcap_{\lambda\in\Lambda}\mathscr{A}_{\lambda} $も$ \sigma $-加法族となる。そこで生成を次のように定義する。

\begin{Def}{}{}
$ \mathscr{G}\subset 2^{S} $について、$ \mathscr{G} $を含む$ S $上の$ \sigma $-加法族全体の交叉を$ \sigma\lbrack \mathscr{G} \rbrack_{S} $、
あるいは単に$ \sigma\lbrack \mathscr{G} \rbrack $で記し、$ \mathscr{G} $により$ X $上で生成された$ \sigma $-加法族と呼ぶ。
このとき$ A\in\sigma\lbrack \mathscr{G} \rbrack $は$ \mathscr{G} $-可測であるという。
\end{Def}

値域の可測性が生成により与えられているとき、写像の可測性は生成元のみを考えれば十分である。

\begin{Prop}{}{}
$ ( S, \mathscr{A} ), ( T, \sigma\lbrack \mathscr{G} \rbrack ) $を可測空間とする。写像$ f\colon S\rightarrow T $について以下は同値である。
\begin{EnumEquiv}
\item $ f $は可測写像である。
\item 任意の$ G\in\mathscr{G} $について$ f^{-1}( G )\in\mathscr{A} $である。
\end{EnumEquiv}
\end{Prop}

\begin{proof}
(証明)上から下は明白。下が成り立つとすると、任意の$ G\in\mathscr{G} $について$ G\in f( \mathscr{A} ) $が成り立つ。
すなわち$ \mathscr{G}\subset f( \mathscr{A} ) $となるが、今$ f( \mathscr{A} ) $は$ \sigma $-加法族であるから、
生成の最小性より$ \sigma\lbrack \mathscr{G} \rbrack\subset f( \mathscr{A} ) $が従う。これは$ f $が可測であることを表している。$ \square $
\end{proof}

\begin{Def}{}{}
$ \mathscr{A}\times T\cup S\times\mathscr{B}=\lbrace A\times T, S\times B : A\in\mathscr{A}, B\in\mathscr{B} \rbrace $により
生成される$ S\times T $上の$ \sigma $-加法族を$ \mathscr{A}\otimes\mathscr{B} $と記し、積$ \sigma $-加法族という。
また組$ ( S\times T, \mathscr{A}\otimes\mathscr{B} ) $を積可測空間という。
\end{Def}

積可測空間が圏$ \mathbf{meas} $における積対象であることを示すために、普遍性を示そう。

\begin{Prop}{}{}
$ ( P, \mathscr{F} ) $を可測空間、$ f\colon P\rightarrow S, g\colon P\rightarrow T $を可測写像とする。
このとき唯一つの可測写像$ h\colon P\rightarrow S\times T $が存在して、図式を可換にする。
\end{Prop}

\begin{proof}
(証明)$ h( x ):=( f( x ), g( x ) ) $と定めればよい。
可測性は先の命題より$ A\in\mathscr{A}, B\in\mathscr{B} $について$ h^{-1}( A\times T ), h^{-1}( S\times B )\in\mathscr{F} $を示せばよいが、例えば
\[ h^{-1}( A\times T )=f^{-1}( A )\cap g^{-1}( T )\in\mathscr{F} \]
より$ f, g $の可測性から従う。

逆に図式を可換にする$ h $はこのような形をしていなければならない。$ \square $
\end{proof}

三つ以上の積について、積が結合的であることや交換できること、普遍性を満たすことも示すことが出来る。この意味で圏$ \mathbf{meas} $は任意の有限積を持つことが分かる。

\begin{comment}生成と積の関係は次の命題が便利である。
\begin{Prop}{}{}
$ \mathscr{G}\subset 2^{S}, \mathscr{H}\subset 2^{T} $に対し、
$ \sigma\lbrack \mathscr{G} \rbrack\otimes\sigma\lbrack \mathscr{H} \rbrack = \sigma\lbrack \mathscr{G}\times T\cup S\times\mathscr{H} \rbrack $が成り立つ。
\end{Prop}

\begin{proof}
(証明)
\[ \sigma\lbrack \mathscr{G} \rbrack\times T\cup S\times\sigma\lbrack \mathscr{H} \rbrack \supset \mathscr{G}\times T\cup S\times\mathscr{H} \]
である。故に生成すれば最小性から
\[ \sigma\lbrack \mathscr{G} \rbrack\otimes\sigma\lbrack \mathscr{H} \rbrack \supset \sigma\lbrack \mathscr{G}\times T\cup S\times\mathscr{H} \rbrack \]
が従う。逆を示すために$ \sigma\lbrack \mathscr{G} \rbrack\times T\subset\sigma\lbrack \mathscr{G}\times T \rbrack $を示す。
射影$ \mathrm{proj}\colon S\times T\rightarrow S $を考えれば
\begin{align*}
\mathrm{proj}\left( \sigma\lbrack \mathscr{G}\times T \rbrack \right) &= \left\lbrace A\subset S : A\times T\in\sigma\lbrack \mathscr{G}\times T \rbrack \right\rbrace \\
&\supset \lbrace A\subset S : A\times T\in\mathscr{G}\times T \rbrace = \mathscr{G}
\end{align*}
となる。故に最小性より$ \mathrm{proj}\left( \sigma\lbrack \mathscr{G}\times T \rbrack \right)\supset\sigma\lbrack \mathscr{G} \rbrack $を得る。つまり
\[ \sigma\lbrack \mathscr{G} \rbrack\times T\subset\sigma\lbrack \mathscr{G}\times T \rbrack\subset\sigma\lbrack \mathscr{G}\times T\cup S\times\mathscr{H} \rbrack \]
が従う。$ S\times\sigma\lbrack \mathscr{H} \rbrack $についても同様だから、
\[ \sigma\lbrack \mathscr{G} \rbrack\times T\cup S\times\sigma\lbrack \mathscr{H} \rbrack \subset \sigma\lbrack \mathscr{G}\times T\cup S\times\mathscr{H} \rbrack \]
が分かる。故に最小性より$ \sigma\lbrack \mathscr{G} \rbrack\otimes\sigma\lbrack \mathscr{H} \rbrack \subset \sigma\lbrack \mathscr{G}\times T\cup S\times\mathscr{H} \rbrack $が従う。$ \square $
\end{proof}\end{comment}




\subsection{測度空間と圏}
測度とは、長さや面積、体積といった概念の抽象化あるいは精密化である。$ \sigma $-加法族と違い簡単には作ることができないため、適切な集合族の上で定義された「前測度」を拡張して構成されることが多い。
より広い意味ではバナッハ空間に値を持つ測度などが定義されるが、ここでは単純に$ \lbrack 0, \infty \rbrack $値のものを考える。

$ S $を空でない集合とする。一般に集合族$ \emptyset\in\mathscr{G}\subset 2^{S} $から$ \lbrack 0, \infty \rbrack $への写像$ \mu $のことを集合函数という。
集合函数の性質について述べた定義はいくつかあるので、適宜紹介していこうと思う。集合$ A, B $は$ A\cap B=\emptyset $のとき互いに素であるという。
\begin{EnumCond}
\item $ \mu( \emptyset )=0 $のとき$ \mu $は正値(positive)であるという。
\item 互いに素な$ A_{1}, \dotsc, A_{m}\in\mathscr{G} $に対し、
\[ A:=\bigsqcup_{i=1}^{m}A_{i}\in\mathscr{G} \Rightarrow \mu( A )=\sum_{i=1}^{m}\mu( A_{i} ) \]
が成り立つとき$ \mu $は有限加法的(finite-additive)であるという。
\item 互いに素な$ \lbrace A_{n} \rbrace_{n\in\mathbb{N}}\subset\mathscr{G} $に対し、
\[ A:=\bigsqcup_{n\in\mathbb{N}} A_{n}\in\mathscr{G} \Rightarrow \mu( A )=\sum_{n\in\mathbb{N}}\mu( A_{n} ) \]
が成り立つとき$ \mu $は可算加法的(countable-additive)であるという。
\end{EnumCond}

\begin{Def}{}{}
$ \mathscr{A}\subset 2^{S} $を$ \sigma $-加法族とする。集合函数$ \mu\colon\mathscr{A}\rightarrow\lbrack 0, \infty \rbrack $が正値かつ可算加法的であるとき、
$ \mu $は$ \mathscr{A} $上の測度(measure)、あるいは$ \mu $を可測空間$ ( S, \mathscr{A} ) $上の測度といい、組$ ( S, \mathscr{A}, \mu ) $を測度空間と呼ぶ。
\end{Def}

例えば次のようなものがある。
\begin{EnumCond}
\item 可測な集合に対して$ 0 $を対応させる写像は測度になる。これを自明測度と呼ぶ。
\item $ A\in\mathscr{A} $が有限集合なら元の個数を対応させ、無限集合のときは$ \infty $を対応させると測度になる。これを数え上げ測度(counting measure)と呼ぶ。
\item $ x\in S $に対して写像$ \delta_{x}\colon\mathscr{A}\rightarrow\lbrack 0, \infty \rbrack $を、$ x\in A $なら$ \delta_{x}( A )=1 $と定め、
$ x\notin A $なら$ \delta_{x}( A )=0 $と定めると測度になる。これをディラック測度(Dirac's measure)と呼ぶ。
\end{EnumCond}

圏$ \mathbf{meas} $に替わるものを考えるとき、測度空間を対象とするものは当然その候補となる。射に関して言うと、圏$ \mathbf{meas} $への忘却函手の存在という観点からすれば、可測写像であることはもちろん前提となるだろう。
ここで問題となるのはその範囲をどうするかである。以下$ ( S, \mathscr{A}, \mu_{S} ), ( T, \mathscr{B}, \mu_{T} ) $を測度空間、$ f\colon S\rightarrow T $を可測写像としよう。
測度の情報をなるべく取り入れるのであれば、測度保存$ \mu_{S}( f^{-1}( B ) )=\mu_{T}( B ) $($ B\in\mathscr{B} $)を仮定すると良く、実際いくつかの論文ではこれが仮定されている。
ただこれでは射が少なすぎる気がしなくもない。そこで測度空間を対象とする3つの圏を定める。
\begin{EnumCond}
\item 任意の$ B\in\mathscr{B} $について$ \mu_{S}( f^{-1}( B ) )=\mu_{T}( B ) $を満たす(測度保存な)可測写像を射とする圏を$ \mathbf{Meas}_{=} $と記す。
\item 任意の$ B\in\mathscr{B} $について$ \mu_{S}( f^{-1}( B ) )\le\mu_{T}( B ) $を満たす(測度増大な)可測写像を射とする圏を$ \mathbf{Meas}_{\le} $と記す。
\item 任意の可測写像を射とする圏を$ \mathbf{Meas} $と記す。
\end{EnumCond}
これらの圏は可測空間の圏$ \mathbf{meas} $よりずっと豊かな構造を持つ。そのため与えられた2つの測度空間が積対象を持つかどうかでさえ簡単には述べることができない。

\end{document}