% ===========================================
% Project:MeasureTheory
% Filename:FinitelyAdditive.tex
% Author:mathmathniconico
% ChangeLog:
%	YYYY/MM/DD
%	2018/05/15	1.0.0	ファイルを分割
% 
% ===========================================

\documentclass[../root.tex]{subfiles}

\begin{document}
この節より用いる集合函数に関する諸定義を述べておこう。$ S $を空でない集合とする。
$ \emptyset\in\mathscr{G}\subset 2^{S} $とし、$ \mu\colon\mathscr{G}\rightarrow\lbrack 0, \infty \rbrack $を集合函数とする。
\begin{EnumCond}
\item 互いに素な$ \lbrace A_{n} \rbrace_{n\in\mathbb{N}}\subset\mathscr{G} $に対し、
\[ A:=\bigsqcup_{n\in\mathbb{N}}A_{n}\in\mathscr{G} \Rightarrow \mu( A )\le\sum_{n\in\mathbb{N}}\mu( A_{n} ) \]
が成り立つとき$ \mu $は弱可算劣加法的(weak countable-subadditive)であるという。
\item 互いに素な$ \lbrace A_{n} \rbrace_{n\in\mathbb{N}}\subset\mathscr{G} $に対し、
\[ A:=\bigsqcup_{n\in\mathbb{N}}A_{n}\in\mathscr{G} \Rightarrow \mu( A )\ge\sum_{n\in\mathbb{N}}\mu( A_{n} ) \]
が成り立つとき$ \mu $は弱可算優加法的(weak countable-superadditive)であるという。
\item $ \lbrace A_{n} \rbrace_{n\in\mathbb{N}}\subset\mathscr{G} $に対し、
\[ A_{n}\nearrow A\in\mathscr{G} \Rightarrow \lim_{n\rightarrow\infty}\mu( A_{n} )=\mu( A ) \]
が成り立つとき$ \mu $は増大列連続であるという。
\item $ \lbrace A_{n} \rbrace_{n\in\mathbb{N}}\subset\mathscr{G} $に対し、
\[ A_{n}\searrow A\in\mathscr{G}, \mu( A_{1} )\lt \infty \Rightarrow \lim_{n\rightarrow\infty}\mu( A_{n} )=\mu( A ) \]
が成り立つとき$ \mu $は減少列連続であるという。
\end{EnumCond}

減少列連続は$ \mu( A_{1} )\lt\infty $を仮定しているので注意すること。




\subsection{有限加法族}
さて測度は外測度より構成できるのだが、その値については簡単な特徴付けがあるくらいで、具体的な値を求めることは難しい。そこで基本となる集合族のクラスを絞り、その上に測度の類似物を与えることで、測度への拡張問題として捉えなおす。
これは例えば$ \mathbb{R} $上の$ a $から$ b $までの区間に対し$ | b-a | $を対応させるようなことを考える。

前節の定理の証明中でも軽く触れたが、有限加法族とは$ \sigma $-加法族の条件のうち可算和を有限和に置き換えたものを満たすクラスのことである。
このクラスにおいても測度の類似物を考えることができ、これを「有限加法族上の前測度」という。前測度という言葉は標語的に用いるもので、より正確を期するには「有限加法族上の正値有限加法的集合函数」と述べるべきだろう。

この節で述べる「有限加法族上の前測度」について成り立つ性質は、もちろん$ \sigma $-加法族上の測度に対しても成り立つ。

\begin{Def}{}{}
集合$ S $において$ \mathscr{A}\subset 2^{S} $が次の3条件を満たすとき、$ \mathscr{A} $は$ S $上の有限加法族であるという。
\begin{EnumCond}
\item $ \emptyset\in\mathscr{A} $である。
\item $ A\in\mathscr{A} $なら$ S\backslash A\in\mathscr{A} $である。
\item $ A, B\in\mathscr{A} $なら$ A\cup B\in\mathscr{A} $である。
\end{EnumCond}

\end{Def}

有限加法族は、有限交叉や差集合で閉じている。つまり$ A, B\in\mathscr{A} $なら$ A\cap B, A\backslash B\in\mathscr{A} $が成り立つ。

有限加法族も歴史的経緯から名前が安定しない。集合代数や集合体とも呼ばれる。

\begin{Rem}{}{}
有限加法族$ \mathscr{A} $上の集合函数$ \mu\colon\mathscr{A}\rightarrow\lbrack 0, \infty \rbrack $に関して、次は同値となる。
\begin{EnumEquiv}
\item $ \mu $は有限加法的である。
\item 互いに素な$ A, B\in\mathscr{A} $について$ \mu( A\sqcup B )=\mu( A )+\mu( B ) $が成り立つ。
\end{EnumEquiv}

また同様に次も同値となる。
\begin{EnumEquiv}
\item $ \mu $は有限劣加法的である。
\item $ A, B\in\mathscr{A} $について$ \mu( A\cup B )\le\mu( A )+\mu( B ) $が成り立つ。
\end{EnumEquiv}
\end{Rem}

この注意は前節の定理の証明中にも暗に使用している。




\subsection{有限加法族上の前測度}
\begin{Def}{}{}
$ \mathscr{A}\subset 2^{S} $を有限加法族とする。集合函数$ \mu\colon\mathscr{A}\rightarrow\lbrack 0, \infty \rbrack $が正値かつ有限加法的であるとき、
$ \mu $は有限加法族$ \mathscr{A} $上の前測度(premeasure)あるいはジョルダン測度という。
\end{Def}

有限加法族上の前測度は単調かつ有限劣加法的である。実際$ A, B\in\mathscr{A} $について
\[ A\subset B \Rightarrow \mu( B )=\mu( A\sqcup( B\backslash A ) )=\mu( A )+\mu( B\backslash A )\ge\mu( A ) \]
である。また
\[ \mu( A\cup B )=\mu( A\sqcup( B\backslash A ) )=\mu( A )+\mu( B\backslash A )\le\mu( A )+\mu( B ) \]
である。

\begin{Lem}{}{}
$ \mu $は有限加法族$ \mathscr{A}\subset 2^{S} $上の前測度とする。以下は同値である。
\begin{EnumEquiv}
\item $ \mu $は可算加法的である。
\item $ \mu $は可算劣加法的である。
\item $ \mu $は弱可算劣加法的である。
\end{EnumEquiv}
\end{Lem}

\begin{proof}
(証明)( a )なら( b )、( b )なら( c )は明白なので( c )なら( a )を示そう。そのためには弱可算優加法的であることを示せば良い。
$ \lbrace A_{n} \rbrace_{n\in\mathbb{N}}\subset\mathscr{A} $は互いに素であるとし、$ A:=\bigsqcup_{n\in\mathbb{N}}A_{n}\in\mathscr{A} $であるとする。
$ \mu $は単調かつ有限加法的なので、$ m $に対して
\[ \mu( A )\ge\mu\left( \bigsqcup_{n=1}^{m} A_{n} \right)=\sum_{n=1}^{m}\mu( A_{n} ) \]
が成り立つ。$ m $は任意なので$ \mu( A )\ge\sum_{n\in\mathbb{N}}\mu( A_{n} ) $が成り立つ。$ \square $
\end{proof}

既に外測度から測度を構成できることを示したが、この定理を用いると有限加法族上の前測度に対し、その拡張となる測度を構成するための必要十分条件を得ることが出来る。

\begin{Thm}{ホップの拡張定理}{}
$ \mu $は有限加法族$ \mathscr{A}\subset 2^{S} $上の前測度とする。以下は同値である。
\begin{EnumEquiv}
\item $ \sigma\lbrack \mathscr{A} \rbrack $上の測度$ \widehat{\mu} $が存在して$ \widehat{\mu}|_{\mathscr{A}}=\mu $を満たす。
つまり$ A\in\mathscr{A} $なら$ \widehat{\mu}( A )=\mu( A ) $が成り立つ。
\item $ \mu $は弱可算劣加法的である。
\end{EnumEquiv}
\end{Thm}

\begin{proof}
(証明)( a )から( b )は明らかなので逆を示す。

$ \mu\colon\mathscr{A}\rightarrow\lbrack 0, \infty \rbrack $から誘導される外測度を$ \widehat{\mu} $と置く。
このとき$ \widehat{\mu} $-可測集合全体$ \mathscr{M}_{\widehat{\mu}} $は$ \sigma $-加法族であり、$ \widehat{\mu} $はその上の測度となる。

まず$ \sigma\lbrack \mathscr{A} \rbrack\subset\mathscr{M}_{\widehat{\mu}} $を示そう。そのためには最小性より$ \mathscr{A}\subset\mathscr{M}_{\widehat{\mu}} $を示せば十分である。
$ A\in\mathscr{A} $及び$ E\subset S $を取る。更に$ E $の被覆$ \mathscr{C}\subset\mathscr{A} $を取れば、$ \mathscr{A} $は有限加法族であるから
$ \lbrace C\cap A : C\in\mathscr{C} \rbrace, \lbrace C\backslash A : C\in\mathscr{C} \rbrace $はそれぞれ$ E\cap A, E\backslash A $の被覆となる。更に$ \mu $は有限加法的だから
\[ \widehat{\mu}( E\cap A )+\widehat{\mu}( E\backslash A ) \le \sum_{C\in\mathscr{C}}( \mu( C\cap A )+\mu( C\backslash A ) ) = \sum_{C\in\mathscr{C}}\mu( C ) \]
となる。右辺の下限を取れば$ \widehat{\mu}( E ) $となるため、$ A $はカラテオドリ可測であることが分かる。

以上により$ \widehat{\mu} $は$ \sigma\lbrack \mathscr{A} \rbrack $上の測度となることが分かった。
最後に$ \mathscr{A} $上での値を見てみよう。まず$ A\in\mathscr{A} $に対して定義より$ \widehat{\mu}( A )\le\mu( A ) $が成り立つ。逆に$ A $の被覆$ \mathscr{C}\subset\mathscr{A} $に対して
\[ A= A\cap\bigcup_{C\in\mathscr{C}}C=\bigcup_{C\in\mathscr{C}}( A\cap C ) \]
となるが、ここで補題より$ \mu $の弱可算劣加法性は可算加法性、特に可算劣加法性と同値であった。故に$ A\cap C\in\mathscr{A} $及び単調性から
\[ \mu( A )=\mu\left( \bigcup_{C\in\mathscr{C}}( A\cap C ) \right)\le\sum_{C\in\mathscr{C}}\mu( A\cap C )\le\sum_{C\in\mathscr{C}}\mu( C ) \]
が従う。右辺の下限を取れば$ \mu( A )\le\widehat{\mu}( A ) $を得る。$ \square $

こうして有限加法族とその上の前測度が与えられたとき、その拡張となる$ \sigma $-加法族とその上の測度が存在することが示された。ちなみに拡張の一意性までは言えない。
\end{proof}

更に次の命題も成り立つ。つまり前測度が測度に拡張されることは、集合族に対するある種の連続性のようなものが成り立つことを意味している。

\begin{Prop}{}{}
$ \mu $は有限加法族$ \mathscr{A}\subset 2^{S} $上の前測度とする。以下は同値である。
\begin{EnumEquiv}
\item $ \mu $は可算加法的である。
\item $ \mu $は増大列連続かつ減少列連続である。
\item $ \lbrace A_{n} \rbrace_{n\in\mathbb{N}}\subset\mathscr{A} $に対し、
\begin{align*}
A_{n}\nearrow A\in\mathscr{A}, \mu( A )=\infty &\Rightarrow \lim_{n\rightarrow\infty}\mu( A_{n} )=\infty \\
A_{n}\searrow \emptyset, \mu( A_{1} )\lt\infty &\Rightarrow \lim_{n\rightarrow\infty}\mu( A_{n} )=0
\end{align*}
が成り立つ。
\end{EnumEquiv}
\end{Prop}

\begin{proof}
(証明)( a )なら( b )が成り立つことを示そう。まず増大列連続であることを示す。$ \lbrace A_{n} \rbrace_{n\in\mathbb{N}}\subset\mathscr{A} $が$ A_{n}\nearrow A\in\mathscr{A} $を満たすとする。
$ A=A_{1}\sqcup\bigsqcup_{n\in\mathbb{N}}( A_{n+1}\backslash A_{n} ) $だから、$ \mu $の可算加法性より$ \mu( A )=\mu( A_{1} )+\sum_{n\in\mathbb{N}}\mu( A_{n+1}\backslash A_{n} ) $となる。
一方$ A_{n}=A_{1}\sqcup\bigsqcup_{k=1}^{n}( A_{k+1}\backslash A_{k} ) $だから
\[ \lim_{n\rightarrow\infty}\mu( A_{n} )=\mu( A_{1} )+\sum_{k=1}^{\infty}\mu( A_{k+1}\backslash A_{k} )=\mu( A ) \]
を得る。次に減少列連続であることを示そう。$ B_{n}\searrow B\in\mathscr{A}, \mu( B_{1} )\lt\infty $とする。
$ B_{n}=B\sqcup\bigsqcup_{k\ge n}( B_{k}\backslash B_{k+1} ) $だから、可算加法性より特に$ n=1 $として$ \infty\gt\mu( B_{1} )\ge\sum_{n\in\mathbb{N}}\mu( B_{n}\backslash B_{n+1} ) $が従う。
つまり$ \lim_{n\rightarrow\infty}\sum_{k\ge n}\mu( B_{k}\backslash B_{k+1} )=0 $だから
\[ \lim_{n\rightarrow\infty}\mu( B_{n} )=\mu( B )+\lim_{n\rightarrow\infty}\sum_{k\ge n}\mu( B_{k}\backslash B_{k+1} )=\mu( B ) \]
を得る。

( b )なら( c )は定義より明らか。

( c )なら( a )が成り立つことを示そう。$ \lbrace A_{n} \rbrace_{n\in\mathbb{N}}\subset\mathscr{A} $は互いに素であるとし、$ A:=\bigsqcup_{n\in\mathbb{N}}A_{n}\in\mathscr{A} $であるとする。
$ B_{m}:=\bigsqcup_{n=1}^{m}A_{n}\in\mathscr{A} $とすると$ B_{m}\nearrow A $である。$ \mu( A )=\infty $のとき仮定より
\[ \mu( A )=\infty=\lim_{m\rightarrow\infty}\mu( B_{m} )=\lim_{m\rightarrow\infty}\sum_{n=1}^{m}\mu( A_{n} )=\sum_{n\in\mathbb{N}}\mu( A_{n} ) \]
となる。$ \mu( A )\lt\infty $のときは$ C_{1}:=A, C_{m+1}:=A\backslash B_{m} $とすると$ C_{m}\searrow\emptyset $かつ$ \mu( C_{1} )=\mu( A )\lt\infty $である。
特に$ B_{m}\subset A $なので$ \mu( C_{m+1} )=\mu( A )-\mu( B_{m} ) $が成り立つ。仮定より$ \lim_{m\rightarrow\infty}\mu( C_{m} )=0 $だから
\[ \sum_{n\in\mathbb{N}}\mu( A_{n} )=\lim_{m\rightarrow\infty}\mu( B_{m} )=\mu( A )-\lim_{m\rightarrow\infty}\mu( C_{m+1} )=\mu( A ) \]
となる。$ \square $
\end{proof}

\end{document}